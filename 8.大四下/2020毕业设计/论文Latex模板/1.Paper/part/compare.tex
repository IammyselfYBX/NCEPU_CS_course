% 本章节介绍与前人的比较
\section{与前人比较}
虽然本实验以前有人在Github上已经实现了,并且发布成项目 \cite{BusyBear} ,但是该实验工程已经四年多没有更新,在根据 README.md 文档重新复现该实验的时候,发现其中很多步骤已经失效了,而且运行很多步骤也会报错,但这些问题并不算什么的,我依旧可以凭借着这几年下来所学的知识进行解决,并成功运行一个演示程序。

比前人改进的地方就是全部软件更新到现在主流版本,修改了一些步骤上遇到的错误,以及解决遇到的一系列问题。

\subsection{收获}
通过该课程设计,让我对Qemu有了更深的理解,再加上对操作系统底层有更深的理解,更深体会到编译工具的熟练使用。

\subsection{不足}
虽然,本文已经可以运行一个程序,但是目前还有如下问题:(1)图形界面,就是用户需要使用图形界面的,这样可以更加方便用户使用,但是目前还没有想清楚如何添加图形界面方法;(2)需要做横向对比,和MIPS,ARM比较,看一下RISC-V的平台会提高多少性能;(3)其实这个也不算标准的移植,成功运行了,但是不代表日后会不会出现函数库,以及依赖的问题;

