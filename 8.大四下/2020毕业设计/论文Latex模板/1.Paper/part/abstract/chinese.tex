% 这是中文摘要
\newpage
\pagenumbering{Roman}
\renewcommand{\abstractname}{\heiti{\Large{摘\ \ \ \ 要}}}
\begin{abstract}
%%%%%%%%%%%%%%%%%%%%%%%%
%想把一些非章节的部分加入目录
\phantomsection
\addcontentsline{toc}{section}{\heiti{\Large{摘 \quad \quad 要}}}    
%%%%%%%%%%%%%%%%%%%%%%%%
~\\
{\large{
    \indent
    RISC-V是一款近年来最为流行的开源指令集架构,而且被广泛应用于各个场景。

    本文是采用Qemu虚拟化模拟仿真出RISC-V指令环境,用于解决跨平台的模型部署和运行问题。并在模拟出来的RISC-V处理器上面进行移植Linux内核、文件系统以及网络协议栈,最后尝试在上面运行RISC-V程序,主要目的是因为本科所学知识理论与实践衔接并不紧密,感觉“纸上得来终觉浅,绝知此事要躬行”,很多内容似懂非懂,一方面还能凭借书本的记忆说上几句,做对几道题,回答上老师的几个问题;另一方面心里又十分清楚地认识到自己学得不是很到位,很多知识内容如果稍微较真一点深入地探讨一下,就会发现自己有很多内容解释不清楚,更谈不上熟练驾驭了,所以本文将对本科所学有关操作系统,计算机组成原理,计算机体系结构还有计算机网络等计算机科学核心课程内容结合RISC-V 指令集和 Linux 操作系统的移植包括一步步完善基本功能来达到对知识的综合理解与回顾。}}


% 下面的是400字,心里比较一下    {\large{正文部分包括:前言、论文主体和结论。要求文章结构严谨,语言流畅,内容正确。前言作为论文的开场白,要以简短的篇幅,说明毕业设计(论文)工作的选题目的和意义、国内外文献综述(或研究动态)以及论文所要研究的正文部分包括:前言、论文主体和结论。要求文章结构严谨,语言流畅,内容正确。前言作为论文的开场白,要以简短的篇幅,说明毕业设计(论文)工作的选题目的和意义、国内外文献综述(或研究动态)以及论文所要研究的正文部分包括:前言、论文主体和结论。要求文章结构严谨,语言流畅,内容正确。前言作为论文的开场白,要以简短的篇幅,说明毕业设计(论文)工作的选题目的和意义、国内外文献综述(或研究动态)以及论文所要研究的正文部分包括:前言、论文主体和结论。要求文章结构严谨,语言流畅,内容正确。前言作为论文的开场白,要以简短的篇幅,说明毕业设计(论文)工作的选题目的和意义、国内外文献综述(或研究动态)以及论文所要研究的}}

%\\ \hspace*{\fill} \\  %换行,用空格填充,再换行,即可实现空出一整行的效果,不需任何环境调整
~\\
\noindent %取消首航缩进
    {\large{\textbf{关键字}:RISC-V,Qemu,linux}}
\end{abstract}
