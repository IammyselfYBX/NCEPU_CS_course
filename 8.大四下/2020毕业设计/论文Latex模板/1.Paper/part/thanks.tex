% 致谢
\newpage
\renewcommand{\abstractname}{\heiti{\huge{致\quad \quad \quad \quad 谢}}}
\begin{abstract}
\phantomsection
\addcontentsline{toc}{section}{\heiti{\Large{致 \quad \quad 谢}}}    
\setlength{\baselineskip}{20pt} %行距20磅
    {\zihao{-4}{ 首先,我非常感谢我的家长,从小到大对我的殷勤付出,最终我才能有今天的收获,是他们的坚持不懈才让本因高考没有考上重点大学的我尽可能没有堕落下去;其次要感谢大学这群功利现实冷酷有没有人情的学生们,他们就是我的好教员,他们用他们冷漠自私让我明白一个道理“落后就要挨打,贫穷就要挨饿,失语就要挨骂”,只有站起来而不是跪下去才会有美好的未来;然后,要感谢毛泽东,邓小平等那些思想领袖,从他们哪里我学到了很多,大学之前没接触过计算机让我刚来的时候挂了不少课也被与预警过,但是面对这样的内外交迫的处境,“速战论”激进思想与“亡国论”的消极思想是不可取的,应该要打持久战,这是一场没有硝烟的战争,大致可划分三个阶段——“战略防御、战略相持、战略反攻”,要团结一切力量,积小胜为大胜,面对帝国主义和资产阶级那种腐朽、堕落和淫乱不堪的作风态度要坚定,最后我在大二结束的暑假就自学完本科全部的内容,并大三顺利地进入战略反攻阶段时,摧枯拉朽般把大一大二欠下的全部课程以及大三大四的课程一并补齐。最后还是要走一条具有自身特色的发展路线,要将理论与实践相结合,鞋子合不合脚自己穿了才知道;最后我要感谢琚赟老师,首先他并不像其他老师那样因为我大一大二基础差而区别对待,其次也没有“好心”地劝过“你学习不好的原因就是因为不努力学习,要抓紧学习”仅仅喊口号并没有提出解决方案,更没有说类似“我当了这么多年老师,你不是我见过聪明的学生”的话,而是采用基层民主制度,让我充分发挥自主的能动性,而不像奴才一样必须被用鞭子催着赶着让去做一件事。

高考刚结束,那时的我们还有梦,还有热情,还有诗和远方;畅想着关于自己,关于未来,关于一场穿越时空的旅行。 而如今我们深夜饮酒, 杯子碰到一起, 都是梦破碎的声音。

    ~\\
\indent   在这四年里 \\
\indent   我知道这个世界很美好,可是我却感受不到!\\
\indent   我知道阳光很暖,风很舒服,可是我更希望你们能被这样的温柔感化!\\    
\indent   这个世间一切都那么美好,希望人与人之间也能像风和太阳那般!\\
\indent   我也知道爱情很美好,可是它不会降临在我身上,所以我希望你们能拥有!\\
\indent   ……\\
\indent   一切的一切我都知道,但是它们都不属于我!\\

\indent   从明天开始,要做一个幸福的人\\ 
\indent   我要闻一闻花的芬芳\\
\indent   我要望一望璀璨的夜空\\
\indent   我要享受一下微风拂面的惬意\\ 
\indent   我要给每一条河每一座山都取一个温暖的名字\\ 
\indent   从明天开始,要告诉每一个关心我的人我的幸福\\ 
\indent   我要祝福每一个陌生人在尘世中过得开心,有光明的前程\\ 
\indent   我愿面朝大海,春暖花开\\ 

}}
\end{abstract}
