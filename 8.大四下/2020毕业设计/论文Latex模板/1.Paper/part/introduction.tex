% 本章节是介绍项目的背景信息
\section{绪论}
\subsection{课题研究背景和意义}
在“中兴事件”刚过没多久,某些国家就开始发动“华为事件”,进行各种制裁,这对我国半导体产业的打击极大,是可忍孰不可忍,放眼整个产业链,我们国家连一个像样的成熟指令集架构都没有,连一个可大规模商用的操作系统都没有,当我们想研究开发产品竟然还需要看别人脸色来得到授权,这是莫大的耻辱。为什么外国人能做出来,我们就做不出来,我们也并不比他们差什么。所以,我觉得我有必要研究这样的问题了,这就是为什么我毕业设计要研究操作系统,指令集这样的内容,帝国主义封锁我们,不让我们发展,但我偏偏就要研究这些偏底层的原理。

首先是选择的指令集——Risc-v,它是一个最近流行基于RISC的指令集架构,它就类似于软件领域开源的linux一样,Risc-V也有在很多领域发挥作用,如果将linux与RiSC-V结合一起来发展的话,我们就可以利用已有的生态,站在巨人们的肩膀上完成自己的目标,而且开源的话可以集思广益,大家一起参与进来,在这个过程中可以发现自己的不足,然后改正,这本身就是一个学习提升的过程,是一个充满意义的事情。


\subsection{国内外研究现状}
美国的加州大学伯克利分校是最早开始RISC-V的开发,并且经过多次优化完善最终形成的第5代精简指令集(RISC),并于2014年发布,它集百家之长吸收了ARM,MIPS,x86和PowerPC的丰富经验,是一个经过模块化的可扩展可以面对不同的应用场景的指令集,它可以通过组合他的不同模块满足不同需求,而且像ARM的传统指令集是不可以进行指令集扩展的,
RISC-V是支持指令扩展的,最关键的还是RISC-V指令架构是没有历史包袱的,
具体体现在以下几个方面,
1.手册页数比较少,与x86和ARM架构的手册达到几千页的数量不同RISC-V架构的手册也就区区不到三百页;2.指令数目上,不同于x86和ARM的指令数那么多并且繁杂,在加上由于历史原因他们不同架构之间的不同分支也相互彼此不兼容。
而RISC-V没有历史包袱,在设计指令集的时候重新从头开始设计,
最终只包含40多个基础指令,并根据这些做为公共部分,再扩展出其他常用的模块指令。实现起来比较简单,在硬件平台上实现也不是很难。

虽然我们的学校的确与美帝之间存在差距,但是这都不算什么,因为就算他们现在比我们发展要超前也这不代表以后永远都会领先我们,如果我们把精力,资源放在这方面,假以时日我们是会迎头赶上的,也可以做出属于自己的产业链。

目前国外SiFive公司服务做的不错,主要业务是提供商业化的针对RISC-V指令集架构设计的应用处理器的IP、相关配套的软件开发平台以及与芯片相关的多种应对策略。
SiFive在2018年注册独立公司SaiFan在中国运行,来服务于中国市场为客服提供服务;
另一家国外公司——Green WavesIoT ,该公司主营功耗低基于RISC-V的用于边缘应用的IP和处理器\cite{Risc-V_development}。

我们之前说了很多关于真正实现国产化指令集的设想,现在RISC-V指令集架构就是这样的一个机遇。
以前的国产芯片领域的开发基本上都需要得到国外公司的授权的才能开发,典型的就是ARM架构,我们的发展花费了大约有十几年的时间来壮大,但重要的是,这些指令集架构是属于外国公司,并不是免费白让我们使用,从根本上来讲是需要他们这些大公司来允许我们使用,如果我国以后使用RISC-V的话,
一方面是可以省去把用来给外国公司交的授权费,用来放在更好的科研攻关上面;
另一方面,外国企业可以随时完全停止合作授权的时代一去不复返了。相比较我们民族大力发展来让自己实现一套自主的指令集架构,成本高,技术难度大,所以这件事本身又没有太大的经济技术应用价值,
因为开发出来一个处理器是要在整个世界上流通的,这样才有可能让各地的软件生态加入建设中来。所以RISC-V的横空出现,就可以很好的处理相应的关系。

国内的芯来科技模仿SiFive做的也不错,提供多个RISC-V IP, 并同时提供蜂鸟E203开发板; 
位于中国台湾的公司——Andes Technology同样推出基于RISC-V的芯片产品; 
在2018年4月,阿里的平头哥收购了中天微并展开了自己的研究,并在2021年发布了目前世界范围内性能最优的RISC-V芯片玄铁910,并且可以与Android OS衔接完美,
这表明目前完善的Android生态环境可以运行在国产RISC-V芯片上了,换句话说软件的生态已经基本解决了。

最后RISC-V对IoT的发展也同样巨大,
现在碎片化是物联网和边缘计算的发展表象,而且以应用为核心的发展方向也逐步成为发展趋势,不同以往传统的围绕着芯片,因为这样的发展方式就会导致以发展模组和应用的公司为核心的,而替代了从前那种以芯片公司为核心的发展方式,传统方式已经面对如今的发展趋势表现出力不从心了。
我们以ARM为例,其特点是不仅价格不亲民每次发布还挺长时间的,如果用于物联网相同的使用场景,就会在市场上趋近相同的的竞争,加上成本高等因素,导致大家都不原意去开发,最终除了少数大企业才能最早买到IP,其他公司只能望尘莫及,这样的后果就是ARM的发展很难去高效的来应对碎片化应用需求。

\subsection{本文研究内容}
本文主要参考孙卫真,刘雪松,朱威浦和向勇老师的《基于RISC-V的计算机系统综合实验设计》\cite{基于RISC-V的计算机系统综合实验设计}上面的内容在Qemu模拟器上将linux试图移植到RISC-V的平台上面,并逐步实现实验环境的搭建,交叉工具链的编译,完善文件系统,安装网络协议栈等现代操作系统应该具备的基本功能,并解决在此过程中遇到的种种挑战,种种报错以及解决方案,最后运行一个RISC-V指令集的简单程序验证一下是否移植成功。

虽然,已经国内已经有人做出了基于RISC-V处理器内核,linus也把riscv-linux合并到Linux的主分支上了,但是我这个毕设的内容就是自己想进行一遍,是验证性的实验,虽然我也知道我现在做的这些事情看来有些微不足道,但是我想既然我已经努力开始做了,就已经具有了加速度,根据$v= v_{0} + at $让子弹再飞一会儿,最终就会获得丰收的成绩。



