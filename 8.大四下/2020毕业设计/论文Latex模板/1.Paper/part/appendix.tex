% 附录

%\CTEXsetup[name={附录,},format={\raggedright\songti\zihao{4}},aftername={\enspace},beforeskip={12bp},afterskip={6bp}]{subsubsection}

\renewcommand\thesection{附录\Alph{subsubsection}}

\newpage
\renewcommand{\abstractname}{\heiti{\zihao{-2} {附\quad \quad \quad \quad 录}}}
\begin{abstract}
\phantomsection
\addcontentsline{toc}{section}{\heiti{\Large{附 \quad \quad 录}}}    

%%%%%%%%%%%%%%%%%%%%%%%%%%%%%%%%%%%%%%%%%
%% 附录A
~\\
\begin{center}
    \heiti{\zihao{-2} {附录A}}
\end{center}
\phantomsection
\addcontentsline{toc}{subsection}{\songti{\large{附录A}}}    
\setlength{\baselineskip}{20pt} %行距20磅
\begin{sloppypar} %文字超出边界的处理方法 
{\zihao{-4} {
    关于Gentoo Linux的部分配置请见:https://www.bilibili.com/video/BV1ny4y1i7G6 的视频演示。

    关于本毕设的实验请见:https://www.bilibili.com/video/BV1QU4y1H7AE。
    
}}

%%%%%%%%%%%%%%%%%%%%%%%%%%%%%%%%%%%%%%%%%
%% 附录B
~\\
\begin{center}
    \heiti{\zihao{-2} {附录B}}
\end{center}
\phantomsection
\addcontentsline{toc}{subsection}{\songti{\large{附录B}}}   
\subsection*{修改USE Flags并安装}
\begin{lstlisting}
# vim /etc/portage/make.conf
QEMU_SOFTMMU_TARGETS="riscv32 risc64"
QEMU_USER_TARGETS="x86_64"

# vim /etc/portage/package.use
app-emulation/qemu qemu_softmmu_targets_arm qemu_softmmu_targets_x86_64
                 qemu_softmmu_targets_sparc
app-emulation/qemu qemu_user_targets_x86_64

% 进行安装
# emerge --ask app-emulation/qemu -y
\end{lstlisting}


\subsection*{安装GNU工具链}
\begin{lstlisting}
mkdir YBX-bishe
cd YBX-bishe/
# 拉取 gnu-toolchain
git clone --recursive https://github.com/riscv/riscv-gnu-toolchain

# 编译生成 RISC-V newlib & Linux toolchains
cd riscv-gnu-toolchain
./configure --prefix=/opt/riscv --enable-multilib
make newlib -j5
make linux -j5
export PATH=$PATH:/opt/riscv/bin
export RISCV=/opt/risc
$
\end{lstlisting}

\subsection*{创建根文件系统}
\begin{lstlisting}
cd ..
git clone https://github.com/michaeljclark/busybear-linux.git
cd busybear-linux
make -j5
\end{lstlisting}

但是这没完,因为busybear会自动帮你下载好 busybox 但是需要自己进行解压和编译

\begin{lstlisting}
CROSS_COMPILE=riscv{{bits}}-unknown-linux-gnu- make menuconfig
CROSS_COMPILE=riscv{{bits}}-unknown-linux-gnu- make
\end{lstlisting}

下面咱们就要制做最小文件系统
\begin{lstlisting}
qemu-img create rootfs.img  1g
mkfs.ext4 rootfs.img
mkdir rootfs
sudo mount -o loop rootfs.img  rootfs
cd rootfs
sudo cp -r ../busyboxsource/_install/* .
sudo mkdir proc sys dev etc etc/init.d
cd etc/init.d/
sudo touch rcS
sudo vi rcS
#!/bin/sh
mount -t proc none /proc
mount -t sysfs none /sys
/sbin/mdev -s

sudo mod +x rcS
sudo umount rootfs
\end{lstlisting}

\subsection*{构建Linux内核}
\begin{lstlisting}
git clone https://github.com/torvalds/linux
cd linux
git checkout v5.4
make ARCH=riscv CROSS_COMPILE=riscv64-unknown-linux-gnu- defconfig
make ARCH=riscv CROSS_COMPILE=riscv64-unknown-linux-gnu-
\end{lstlisting}

\subsection*{制作BootLoader——BBL(Berkeley Boot Loader)}
\begin{lstlisting}
cd..
git clone https://github.com/riscv/riscv-pk.git
cd riscv-pk
mkdir build
cd build
../configure \
> --enable-logo \
> --host=riscv64-unknown-elf \
> --with-payload=../../riscv-linux/vmlinux

make -j8
\end{lstlisting}

\subsection*{编写各种所需脚本}
\begin{lstlisting}
cd ..
mkdir Running
cd Running
\end{lstlisting}
创建KVM启动脚本
\begin{lstlisting}
vim open_kvm
#!/bin/sh
sudo /etc/init.d/libvirtd start
sudo virsh net-start default  #开启网络服务
\end{lstlisting}

创建KVM关闭脚本
\begin{lstlisting}
sudo virsh net-destory default
sudo /etc/init.d/libvirtd stop
\end{lstlisting}


创建网络启动脚本
\begin{lstlisting}
#!/bin/sh
brctl addif virbr0 $1
ifconfig $1 up
\end{lstlisting}

创建网络关闭脚本
\begin{lstlisting}
#!/bin/sh
ifconfig $1 down
brctl delif virbr0 $1
\end{lstlisting}

创建程序运行脚本
\begin{lstlisting}
#!/bin/sh
# QEMU 5.2以后.模拟器内部集成了OpenSBI
sudo qemu-system-riscv64 \
  -nographic -machine virt \
  -m 1024M \
  -kernel bbl \
  -kernel ~/Documents/Risc-v/busybear-linux/build/linux-5.0/arch
                  /riscv/boot/Image \
  -drive file=busybear.bin,format=raw,id=hd0 \
  -device virtio-blk-device,drive=hd0 \
  -device virtio-net-device,netdev=net0 \
  -netdev type=tap,script=./ifup.sh,downscript=./ifdown.sh,id=net0 \
  -append "root=/dev/vda ro console=ttyS0"
\end{lstlisting}

\subsection*{运行}
\begin{lstlisting}
./Running.sh
\end{lstlisting}


\end{sloppypar}
\end{abstract}
